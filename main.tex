\documentclass[12pt, a4paper]{article}
\usepackage[T1]{fontenc}
\usepackage[utf8]{inputenc}
\usepackage[english]{babel}

\usepackage{microtype}
\usepackage{amsmath,amsfonts,amsthm}
\usepackage{graphicx}
\usepackage{url}
\usepackage{geometry}
\usepackage{hyperref}
\usepackage{fancyhdr}
\usepackage{enumitem}
\usepackage{tabularx}
\usepackage{mathtools}
\usepackage{csquotes}
\usepackage[style=apa]{biblatex}
\addbibresource{ref.bib}
% Adjust margins here

\geometry{left=3cm, right=3cm, top=3cm, bottom=3cm, headheight=15pt}
\addtolength{\topmargin}{-2.5pt}
% Increase bottom margin to lower page numbering

\pagestyle{fancy}
\fancyhf{} % clear all header and footer fields
\fancyhead[L]{MATH 323: Actuarial Mathematics I  } % left header
\fancyhead[R]{Homework Report 1} % right header
\fancyfoot[C]{\thepage} % center footer
\renewcommand{\headrulewidth}{0.4pt} % header rule width
\renewcommand{\footrulewidth}{0.4pt} % footer rule width

\begin{document}

\begin{titlepage}
    \centering
    
    \vspace*{0.5cm}
    
    {\Large\bfseries MATH 323: Actuarial Mathematics I\par}
    
    \vspace{1cm}
    
    {\large Homework Report 1\par}
    
    \vspace{0.5cm}
    
    {\today\par}
    
    \vspace{1pt}
    
    \includegraphics[width=0.3\textwidth]{NU-logo.png}\\
    \includegraphics[width=0.15\textwidth]{sosah-logo.png}

    \vspace{0.5cm}
    
    Submitted for {\bf MATH 323: Actuarial Mathematics I} at the School of Sciences and Humanities, Department of Mathematics, Nazarbayev University
    
    \vspace{0.5cm}
    
    {\large Student Name:\par}
    \begin{itemize}[leftmargin=5cm,rightmargin=4cm]
        \item  Aigerim Tursynbekkyzy - ID: 202043550
    \end{itemize}

    \vspace{0.5cm}
    
    \flushleft{
  Subject Area: {\bf Theory of Interest} \\
  Description: {\bf Homework Problems in Chapter 1 and Chapter 22 \\
  Course Instructor : {\bf Dongming Wei} \\
    }
    
    \vspace{0.5cm}
    
    {\footnotesize In submitting this work we are indicating
    that we have read the University's Academic Integrity Policy. We
    declare that all material in this assessment is our own work except
    where there is clear acknowledgment and reference to the work of
    others.\par}
\end{titlepage}}
The following is the used as solutions samples for each problem:
\newpage
\section*{Problems}
\subsection*{Question 19, Chapter 1  (\cite{toi3rd})}
\noindent It is known that an investment of \$500 will increase to \$4000 at the end of 30 years. Find the sum of the present values of three payments of \$10,000 each which will occur at the end of 20, 40, and 60 years.

\subsection*{Question 24, Chapter 1  (\cite{toi3rd})}
\noindent Show that
\[
\frac{d^3}{(1 - d)^2} = \frac{(i - d)^2}{1 - v}.
\]

\subsection*{Question 26, Chapter 1  (\cite{toi3rd})}
\noindent
\begin{enumerate}
    \item[(a)] Express \( d^{(4)} \) as a function of \( i^{(3)} \).
    \item[(b)] Express \( i^{(6)} \) as a function of \( d^{(2)} \).
\end{enumerate}

\subsection*{Question 27, Chapter 1  (\cite{toi3rd})}
\noindent
\begin{enumerate}
    \item[(a)] Show that \( i^{(m)} = d^{(m)} (1 + i)^{1/m} \).
    \item[(b)] Verbally interpret the result obtained in (a).
\end{enumerate}

\subsection*{Question 28, Chapter 1  (\cite{toi3rd})}
\noindent Find the accumulated value of \$100 at the end of two years:
\begin{enumerate}
    \item[(a)] If the nominal annual rate of interest is 6\% convertible quarterly.
    \item[(b)] If the nominal annual rate of discount is 6\% convertible once every four years.
\end{enumerate}

\subsection*{Question 34, Chapter 1  (\cite{toi3rd})}
\noindent Fund A accumulates at a simple interest rate of 10\%. Fund B accumulates at a simple discount rate of 5\%. Find the point in time at which the forces of interest on the two funds are equal.

\subsection*{Question 35, Chapter 1  (\cite{toi3rd})}
\noindent An investment is made for one year in a fund whose accumulation function is a second degree polynomial. The nominal rate of interest earned during the first half of the year is 5\% convertible semiannually. The effective rate of interest earned for the entire year is 7\%. Find \( \delta_s \).

\subsection*{Question 37, Chapter 1  (\cite{toi3rd})}
\noindent Find the level effective rate of interest over a three-year period which is equivalent to an effective rate of discount of 8\% the first year, 7\% the second year, and 6\% the third year.

\subsection*{Question 46, Chapter 1  (\cite{toi3rd})}
\noindent You are given \( \delta_t = \frac{2}{t-1} \) for \( 2 \leq t \leq 10 \). For any one-year interval between times \( n \) and \( n + 1 \), with \( n = 2, 3, \dots, 9 \), calculate the equivalent \( d^{(2)} \).

\subsection*{Question 49, Chapter 1  (\cite{toi3rd})}
\noindent Find the following derivatives:
\begin{enumerate}
    \item[(a)] \( \frac{dd}{di} \)
    \item[(b)] \( \frac{d\delta}{di} \)
    \item[(c)] \( \frac{d\delta}{dv} \)
    \item[(d)] \( \frac{dd}{d\delta} \)
\end{enumerate}


\subsection*{Question 2, Chapter 2 (\cite{toi3rd})}

\noindent You have an inactive credit card with a \$1000 outstanding unpaid balance. This particular credit card charges interest at the rate of 18\% compounded monthly. You are able to make a payment of \$200 one month from today and \$300 two months from today. Find the amount that you will have to pay three months from today to completely pay off this credit card debt. (Note: Work this problem with an equation of value. You will learn an alternative approach for this type of problem in Chapter 5.)

\bigskip


\subsection*{Question 3, Chapter 2 (\cite{toi3rd})}

\noindent At a certain interest rate the present value of the following two payment patterns are equal:

\[
(i) \quad 200 \text{ at the end of 5 years plus } 500 \text{ at the end of 10 years.}
\]

\[
(ii) \quad 400.94 \text{ at the end of 5 years.}
\]

At the same interest rate, \$100 invested now plus \$120 invested at the end of 5 years will accumulate to \( P \) at the end of 10 years. Calculate \( P \).

\bigskip


\subsection*{Question 5, Chapter 2 (\cite{toi3rd})}

\noindent Whereas the choice of a comparison date has no effect on the answer obtained with compound interest, the same cannot be said of simple interest. Find the amount to be paid at the end of 10 years which is equivalent to two payments of \$100 each, the first to be paid immediately and the second to be paid at the end of 5 years. Assume 5\% simple interest is earned from the date each payment is made and use a comparison date of:

\[
\text{a) The end of 10 years.}
\]
\[
\text{b) The end of 15 years.}
\]

\bigskip



\subsection*{Question 10, Chapter 2 (\cite{toi3rd})}

\noindent You are asked to develop a rule of \( n \) to approximate how long it takes money to triple. Find \( n \), where \( n \) is a positive integer.

\bigskip



\subsection*{Question 11, Chapter 2 (\cite{toi3rd})}

\noindent A deposits 10 today and another 30 in five years into a fund paying simple interest of 11\% per year. B will make the same two deposits, but the 10 will be deposited \( n \) years from today and the 30 will be deposited \( 2n \) years from today. B’s deposits earn an annual effective rate of 9.15\%. At the end of 10 years, the accumulated value of B’s deposits equals the accumulated value of A’s deposits. Calculate \( n \).

\bigskip


\subsection*{Question 13, Chapter 2 (\cite{toi3rd})}

\noindent Find the nominal rate of interest convertible semiannually at which the accumulated value of \$1000 at the end of 15 years is \$3000.

\bigskip



\subsection*{Question 16, Chapter 2 (\cite{toi3rd})}

\noindent It is known that an investment of \$1000 will accumulate to \$1825 at the end of 10 years. If it is assumed that the investment earns simple interest at rate \( i \) during the 1st year, \( 2i \) during the 2nd year, ..., \( 10i \) during the 10th year, find \( i \).

\bigskip



\subsection*{Question 17, Chapter 2 (\cite{toi3rd})}

\noindent It is known that an amount of money will double itself in 10 years at a varying force of interest \( \delta_t = kt \). Find an expression for \( k \).

\bigskip



\subsection*{Question 18, Chapter 2 (\cite{toi3rd})}

\noindent The sum of the accumulated value of 1 at the end of three years at a certain effective rate of interest \( i \), and the present value of 1 to be paid at the end of three years at an effective rate of discount numerically equal to \( i \) is 2.0096. Find the rate \( i \).

\bigskip


\subsection*{Question 20, Chapter 2 (\cite{toi3rd})}

\noindent A sum of \$10,000 is invested for the months of July and August at 6\% simple interest. Find the amount of interest earned:

\[
\text{a) Assuming exact simple interest.}
\]
\[
\text{b) Assuming ordinary simple interest.}
\]
\[
\text{c) Assuming the Banker’s Rule.}
\]

    \bigskip


\subsection*{Question 21, Chapter 2 (\cite{toi3rd})}
\[
\text{a) Show that the Banker’s Rule is always more favorable to the lender than is exact simple interest.}
\]
\[
\text{b) Show that the Banker’s Rule is usually more favorable to the lender than is ordinary simple interest.}
\]
\[
\text{c) Find a counterexample in (b) for which the opposite relationship holds.}
\]

\newpage

\section*{Solutions}

\subsection*{Question 19, Chapter 1  (\cite{toi3rd})}

\[
500(1+i)^{30} = 4000 \quad \Rightarrow \quad (1+i)^{30} = 8
\]
\[
PV = \frac{10000}{(1+i)^{20}} + \frac{10000}{(1+i)^{40}} + \frac{10000}{(1+i)^{60}}
\]
\[
v = \frac{1}{1+i}, \quad v^{30} = \tfrac{1}{8}
\]
\[
PV = 10000\left(v^{20} + v^{40} + v^{60}\right)
\]
\[
= 10000\left(\left(\tfrac{1}{8}\right)^{2/3} + \left(\tfrac{1}{8}\right)^{4/3} + \tfrac{1}{64}\right)
\]
\[
= 10000\left(\tfrac{1}{4} + \tfrac{1}{16} + \tfrac{1}{64}\right)
\]
\[
= 10000\left(0.25 + 0.0625 + 0.015625\right) = 3281.25
\]
\[
\boxed{3281.25}
\]

\subsection*{Question 24, Chapter 1  (\cite{toi3rd})}

We want to show that
\[
\frac{d^3}{(1-d)^2} = \frac{(i-d)^2}{1-v}.
\]
Recall the standard relationships:
\[
d = \frac{i}{1+i}, \quad v = \frac{1}{1+i}, \quad 1-d = v.
\]
\textbf{LHS:}
\[
\frac{d^3}{(1-d)^2} = \frac{d^3}{v^2}.
\]
Since
\[
d = \frac{i}{1+i}, \quad d^3 = \frac{i^3}{(1+i)^3}, \quad v^2 = \frac{1}{(1+i)^2},
\]
we get
\[
\frac{d^3}{v^2} = \frac{\tfrac{i^3}{(1+i)^3}}{\tfrac{1}{(1+i)^2}}
= \frac{i^3}{(1+i)^3}(1+i)^2 = \frac{i^3}{1+i}.
\]
\textbf{RHS:}
\[
\frac{(i-d)^2}{1-v}.
\]
Now
\[
i-d = i - \frac{i}{1+i} = \frac{i^2}{1+i},
\]
so
\[
(i-d)^2 = \frac{i^4}{(1+i)^2}.
\]
Also
\[
1-v = 1 - \frac{1}{1+i} = \frac{i}{1+i}.
\]
Thus
\[
\frac{(i-d)^2}{1-v} = \frac{\tfrac{i^4}{(1+i)^2}}{\tfrac{i}{1+i}}
= \frac{i^4}{(1+i)^2}\cdot \frac{1+i}{i}
= \frac{i^3}{1+i}.
\]
Since both sides equal \(\tfrac{i^3}{1+i}\), the identity is proven:
\[
\frac{d^3}{(1-d)^2} = \frac{(i-d)^2}{1-v}.
\]

\subsection*{Question 26, Chapter 1  (\cite{toi3rd})}

\textbf{(a)} We know
\[
d^{(m)} = 1 - (1+i)^{-1/m}, \quad 
1+i^{(m)} = (1+i)^{1/m}.
\]

From \(i^{(3)}\):
\[
1+i^{(3)} = (1+i)^{1/3} \quad \Rightarrow \quad 1+i = (1+i^{(3)})^3.
\]

Thus
\[
d^{(4)} = 1 - (1+i)^{-1/4} = 1 - (1+i^{(3)})^{-3/4}.
\]

\[
\boxed{d^{(4)} = 1 - (1+i^{(3)})^{-3/4}}
\]

\bigskip

\textbf{(b)} From the discount relation:
\[
1-d^{(m)} = (1+i)^{-1/m}.
\]

For \(m=2\):
\[
1-d^{(2)} = (1+i)^{-1/2} \quad \Rightarrow \quad 1+i = \left(\frac{1}{1-d^{(2)}}\right)^2.
\]

Hence
\[
1+i^{(6)} = (1+i)^{1/6} = \left(\frac{1}{1-d^{(2)}}\right)^{1/3}.
\]

\[
i^{(6)} = (1-d^{(2)})^{-1/3} - 1.
\]

\[
\boxed{i^{(6)} = (1-d^{(2)})^{-1/3} - 1}
\]

\subsection*{Question 27, Chapter 1  (\cite{toi3rd})}

\textbf{27(a)} Show that
\[
i^{(m)} = d^{(m)} (1+i)^{1/m}.
\]

We know
\[
i^{(m)} = m \big( (1+i)^{1/m} - 1 \big),
\quad 
d^{(m)} = m \big( 1 - (1+i)^{-1/m} \big).
\]

Thus
\[
d^{(m)} = m \cdot \frac{(1+i)^{1/m} - 1}{(1+i)^{1/m}}.
\]

Multiplying both sides by \((1+i)^{1/m}\):
\[
d^{(m)} (1+i)^{1/m} 
= m \cdot \frac{(1+i)^{1/m} - 1}{(1+i)^{1/m}} \cdot (1+i)^{1/m}
= m \big( (1+i)^{1/m} - 1 \big).
\]

\[
\boxed{i^{(m)} = d^{(m)} (1+i)^{1/m}}
\]

\bigskip
\textbf{27(b)} \\
This result shows that the nominal interest rate convertible \(m\) times per year
equals the nominal discount rate convertible \(m\) times per year,
multiplied by the accumulation factor for a period of length \(1/m\) year.

\subsection*{Question 28, Chapter 1  (\cite{toi3rd})}

\textbf{28.}
\textbf{(a)} Nominal annual rate of interest is 6\% convertible quarterly.  

\[
i^{(4)} = 0.06, 
\quad i_{\text{quarter}} = \frac{0.06}{4} = 0.015
\]

Number of quarters in 2 years: \( 2 \times 4 = 8 \).  

\[
FV = 100(1+0.015)^8 \approx 100(1.12674) = 112.67
\]

\[
\boxed{112.67}
\]

\bigskip
\textbf{(b)} Nominal annual rate of discount is 6\% convertible once every 4 years.  

\[
d^{(1)} = 0.06, 
\quad v_4 = 1-d = 0.94
\]

\[
(1+i)^4 = \frac{1}{0.94}, 
\quad 1+i = \left(\frac{1}{0.94}\right)^{1/4}
\]

After 2 years:
\[
FV = 100 (1+i)^2 = 100 \left(\frac{1}{0.94}\right)^{1/2} \approx 103.16
\]

\[
\boxed{103.16}
\]

\subsection*{Question 34, Chapter 1  (\cite{toi3rd})}

\textbf{34.} Fund A accumulates at a simple interest rate of 10\%.  
Fund B accumulates at a simple discount rate of 5\%.  
Find the point in time at which the forces of interest on the two funds are equal.

\bigskip
For Fund A (simple interest, \(i = 0.10\)):  
\[
a_A(t) = 1 + 0.10t, 
\quad \delta_A(t) = \frac{a_A'(t)}{a_A(t)} 
= \frac{0.10}{1+0.10t}.
\]

For Fund B (simple discount, \(d = 0.05\)):  
\[
a_B(t) = \frac{1}{1-0.05t}, 
\quad \delta_B(t) = \frac{a_B'(t)}{a_B(t)} 
= \frac{0.05}{1-0.05t}.
\]

Set equal:
\[
\frac{0.10}{1+0.10t} = \frac{0.05}{1-0.05t}.
\]

Cross multiply:
\[
0.10(1-0.05t) = 0.05(1+0.10t),
\]
\[
0.10 - 0.005t = 0.05 + 0.005t,
\]
\[
0.05 = 0.01t \quad \Rightarrow \quad t = 5.
\]

\[
\boxed{t = 5 \text{ years}}
\]

\subsection*{Question 35, Chapter 1  (\cite{toi3rd})}

\textbf{35.} An investment is made for one year in a fund whose accumulation function 
is a second-degree polynomial. The nominal rate of interest earned during the first half 
of the year is 5\% convertible semiannually. The effective rate of interest earned for the entire year is 7\%. Find \(\delta_s\).

\bigskip
Let the accumulation function be
\[
a(t) = 1 + \alpha t + \beta t^2.
\]

At \(t=0\), \(a(0)=1\).  
At \(t=1\), effective interest = 7\%:
\[
a(1) = 1.07 = 1 + \alpha + \beta \quad \Rightarrow \quad \alpha + \beta = 0.07.
\]

At \(t=\tfrac{1}{2}\), the effective interest is 
\[
i^{(2)} = \frac{0.05}{2} = 0.025,
\quad a\!\left(\tfrac{1}{2}\right) = 1.025.
\]

So:
\[
1 + \frac{\alpha}{2} + \frac{\beta}{4} = 1.025 
\quad \Rightarrow \quad \frac{\alpha}{2} + \frac{\beta}{4} = 0.025.
\]

Multiply by 4:
\[
2\alpha + \beta = 0.1.
\]

Now system:
\[
\alpha + \beta = 0.07, \quad 2\alpha + \beta = 0.1.
\]

Subtract:
\[
\alpha = 0.03, \quad \beta = 0.04.
\]

Thus
\[
a(t) = 1 + 0.03t + 0.04t^2.
\]

Force of interest:
\[
\delta(t) = \frac{a'(t)}{a(t)} 
= \frac{0.03 + 0.08t}{1+0.03t+0.04t^2}.
\]

At \(t=s\):
\[
\boxed{\delta_s = \frac{0.03 + 0.08s}{1+0.03s+0.04s^2}}
\]


\subsection*{Question 37, Chapter 1  (\cite{toi3rd})}

\textbf{37.} Find the level effective rate of interest over a three-year period 
which is equivalent to an effective rate of discount of 8\% the first year, 
7\% the second year, and 6\% the third year.

\bigskip
We use the relation between discount and interest:
\[
1-d = v = \frac{1}{1+i}, 
\quad \Rightarrow \quad 1+i = \frac{1}{1-d}.
\]

For each year:
\[
1+i_1 = \frac{1}{1-0.08} = \frac{1}{0.92} = 1.086956,
\]
\[
1+i_2 = \frac{1}{1-0.07} = \frac{1}{0.93} = 1.075269,
\]
\[
1+i_3 = \frac{1}{1-0.06} = \frac{1}{0.94} = 1.063830.
\]

The three-year accumulation factor is
\[
A = (1+i_1)(1+i_2)(1+i_3) \approx 1.2477.
\]

We want a level effective annual rate \(i\) such that
\[
(1+i)^3 = A,
\quad \Rightarrow \quad i = A^{1/3} - 1.
\]

\[
i \approx (1.2477)^{1/3} - 1 \approx 0.0764.
\]

\[
\boxed{i \approx 7.64\%}
\]

\subsection*{Question 46, Chapter 1  (\cite{toi3rd})}    

\textbf{46.} You are given 
\[
\delta_t = \frac{2}{t-1}, \quad 2 \leq t \leq 10.
\]

For the interval \([n, n+1]\), the effective accumulation is
\[
1+i_n = \exp\!\left(\int_n^{n+1} \delta_t \, dt\right).
\]

\[
\int_n^{n+1} \frac{2}{t-1} \, dt 
= 2 \ln\!\left(\frac{n}{n-1}\right),
\]
\[
1+i_n = \left(\frac{n}{n-1}\right)^2.
\]

Now, relation between effective interest and nominal discount:
\[
1+i_n = \left(\frac{1}{1-d^{(2)}/2}\right)^2.
\]

\[
\sqrt{1+i_n} = \frac{1}{1-d^{(2)}/2},
\quad d^{(2)} = 2\left(1 - \frac{1}{\sqrt{1+i_n}}\right).
\]

Since 
\[
1+i_n = \left(\frac{n}{n-1}\right)^2,
\quad \sqrt{1+i_n} = \frac{n}{n-1},
\]
we get
\[
d^{(2)} = 2\left(1 - \frac{n-1}{n}\right) = \frac{2}{n}.
\]

\[
\boxed{d^{(2)} = \tfrac{2}{n}, \quad n=2,3,\dots,9.}
\]

\subsection*{Question 49, Chapter 1  (\cite{toi3rd})}

\textbf{49.} Find the following derivatives.

\bigskip
\textbf{(a)} 
\[
d = \frac{i}{1+i}, 
\quad \frac{dd}{di} = \frac{1}{(1+i)^2}.
\]

\bigskip
\textbf{(b)} 
\[
\delta = \ln(1+i),
\quad \frac{d\delta}{di} = \frac{1}{1+i}.
\]

\textbf{(c)} 
\[
v = \frac{1}{1+i}, \quad 
\delta = \ln(1+i) = -\ln v,
\]
\[
\frac{d\delta}{dv} = -\frac{1}{v}.
\]

\textbf{(d)} 
\[
\frac{dd}{d\delta} 
= \frac{\tfrac{dd}{di}}{\tfrac{d\delta}{di}}
= \frac{\tfrac{1}{(1+i)^2}}{\tfrac{1}{1+i}}
= \frac{1}{1+i} = v.
\]

\[
\boxed{\frac{dd}{d\delta} = v}
\]

\subsection*{Question 2, Chapter 2 (\cite{toi3rd})}

\textbf{2.} Outstanding debt: \$1000, monthly interest \(= 0.18/12 = 0.015\).  
Payments: \$200 at month 1, \$300 at month 2, and \(X\) at month 3.

\[
1000(1.015)^3 = 200(1.015)^2 + 300(1.015) + X
\]

\[
X = 1000(1.015)^3 - 200(1.015)^2 - 300(1.015)
\]

\[
X \approx 535.13
\]

\[
\boxed{X \approx 535.13}
\]


\subsection*{Question 3, Chapter 2 (\cite{toi3rd})}

\textbf{3.} Equating present values:

\[
\frac{200}{(1+i)^5} + \frac{500}{(1+i)^{10}} 
= \frac{400.94}{(1+i)^5}
\]

\[
200(1+i)^5 + 500 = 400.94(1+i)^5
\]

\[
200.94(1+i)^5 = 500, 
\quad (1+i)^5 \approx 2.487, 
\quad i \approx 20\%.
\]

Now,
\[
P = 100(1.2)^{10} + 120(1.2)^5
\]

\[
P \approx 100(6.1917) + 120(2.4883) = 917.77
\]

\[
\boxed{P \approx 917.77}
\]


\subsection*{Question 5, Chapter 2 (\cite{toi3rd})}

\textbf{5.} Two payments of \$100: one at time 0 and one at time 5.  
Simple interest rate: 5\%.

\bigskip
\textbf{(a) Comparison date: end of 10 years}

First payment accumulated for 10 years:
\[
100(1+0.05 \times 10) = 150
\]

Second payment accumulated for 5 years:
\[
100(1+0.05 \times 5) = 125
\]

Total at year 10:
\[
150 + 125 = 275
\]

\[
\boxed{275}
\]

\bigskip
\textbf{(b) Comparison date: end of 15 years}

First payment accumulated for 15 years:
\[
100(1+0.05 \times 15) = 175
\]

Second payment accumulated for 10 years:
\[
100(1+0.05 \times 10) = 150
\]

Total at year 15:
\[
175 + 150 = 325
\]

\[
\boxed{325}
\]


\subsection*{Question 10, Chapter 2 (\cite{toi3rd})}

\textbf{10.} Develop a rule of \(n\) to approximate how long it takes money to triple.

We want
\[
(1+i)^n = 3.
\]

Taking logs:
\[
n = \frac{\ln 3}{\ln(1+i)}.
\]

Using the \emph{rule of \(n\)} approximation:
\[
n \approx \frac{\ln 3}{i}.
\]

Numerical constant:
\[
\ln 3 \approx 1.099.
\]

So:
\[
\boxed{n \approx \frac{1.099}{i}}
\]

\subsection*{Question 11, Chapter 2 (\cite{toi3rd})}

\textbf{11.}  
A deposits 10 at \(t=0\) and 30 at \(t=5\) into a fund paying simple interest of 11\%.  

\bigskip
\textbf{A’s deposits:}

Value of 10 after 10 years:
\[
10 \left(1 + 0.11 \times 10\right) = 10(2.1) = 21.
\]

Value of 30 after 5 years (from \(t=5\) to \(t=10\)):
\[
30 \left(1 + 0.11 \times 5\right) = 30(1.55) = 46.5.
\]

Total at year 10:
\[
AV_A = 21 + 46.5 = 67.5.
\]

\bigskip
\textbf{B’s deposits:}  
10 deposited at \(t=n\), accumulates to:
\[
10(1.0915)^{10-n}.
\]

30 deposited at \(t=2n\), accumulates to:
\[
30(1.0915)^{10-2n}.
\]

Total at year 10:
\[
AV_B = 10(1.0915)^{10-n} + 30(1.0915)^{10-2n}.
\]

\bigskip
\textbf{Equation:}
\[
10(1.0915)^{10-n} + 30(1.0915)^{10-2n} = 67.5.
\]

Solve numerically for \(n\):
\[
n \approx 7.
\]

\[
\boxed{n \approx 7}
\]

\subsection*{Question 13, Chapter 2 (\cite{toi3rd})}

\textbf{13.} Find the nominal rate of interest convertible semiannually at which 
the accumulated value of \$1000 at the end of 15 years is \$3000.

\bigskip
We require
\[
1000(1+j^{(2)}/2)^{30} = 3000.
\]

\[
(1+j^{(2)}/2)^{30} = 3,
\quad 1+\frac{j^{(2)}}{2} = 3^{1/30}.
\]

\[
j^{(2)} = 2\left(3^{1/30} - 1\right).
\]

Numerically,
\[
3^{1/30} \approx 1.03728,
\quad j^{(2)} \approx 2(0.03728) = 0.07456.
\]

\[
\boxed{j^{(2)} \approx 7.46\%}
\]

\subsection*{Question 16, Chapter 2 (\cite{toi3rd})}

\textbf{16.} An investment of \$1000 accumulates to \$1825 at the end of 10 years.  
It earns simple interest at rates \(i, 2i, \dots, 10i\) for each respective year.  

\bigskip
Total accumulated value:
\[
1000\left(1 + i + 2i + \cdots + 10i\right).
\]

\[
= 1000\left(1 + i(1+2+\cdots+10)\right).
\]

\[
= 1000\left(1 + i\cdot 55\right).
\]

Given:
\[
1000(1+55i) = 1825.
\]

\[
1+55i = 1.825, \quad 55i = 0.825, \quad i = \frac{0.825}{55}.
\]

\[
i = 0.015 \quad \Rightarrow \quad \boxed{1.5\%}
\]

\subsection*{Question 17, Chapter 2 (\cite{toi3rd})}

\textbf{17.} An amount of money doubles in 10 years with varying force of interest 
\(\delta_t = kt\).  

\bigskip
Accumulated value factor:
\[
\exp\!\left(\int_0^{10} kt \, dt \right).
\]

\[
= \exp\!\left( k \cdot \frac{10^2}{2} \right) = \exp(50k).
\]

We need:
\[
\exp(50k) = 2.
\]

\[
50k = \ln 2, \quad k = \frac{\ln 2}{50}.
\]

\[
\boxed{k = \frac{\ln 2}{50}}
\]

\subsection*{Question 18, Chapter 2 (\cite{toi3rd})}

\textbf{18.} The sum of accumulated value of 1 at the end of 3 years at effective rate \(i\) 
and the present value of 1 due at end of 3 years at effective discount \(d=i\) is 2.0096.  

\bigskip
Accumulated value:
\[
(1+i)^3.
\]

Present value at discount rate \(d=i\):
\[
\frac{1}{1+3i}.
\]

Equation:
\[
(1+i)^3 + \frac{1}{1+3i} = 2.0096.
\]

Solve numerically:  
Approximate solution:
\[
i \approx 0.02.
\]

\[
\boxed{i \approx 2\%}
\]

\subsection*{Question 20, Chapter 2 (\cite{toi3rd})}

\textbf{20.} A sum of \$10{,}000 is invested for the months of July and August at 6\% simple interest.  

\bigskip
\textbf{(a) Exact simple interest}  
Exact time = 62 days (July 31 + August 31).  
Year = 365 days.  

\[
I = P \cdot i \cdot \frac{t}{365}
= 10000 \cdot 0.06 \cdot \frac{62}{365}.
\]

\[
I \approx 101.92.
\]

\[
\boxed{101.92}
\]

\bigskip
\textbf{(b) Ordinary simple interest}  
Ordinary year = 360 days.  
\[
I = 10000 \cdot 0.06 \cdot \frac{62}{360}.
\]

\[
I \approx 103.33.
\]

\[
\boxed{103.33}
\]

\bigskip
\textbf{(c) Banker's Rule}  
Use exact days with ordinary year:  

\[
I = 10000 \cdot 0.06 \cdot \frac{62}{360}.
\]

\[
I \approx 103.33.
\]

\[
\boxed{103.33}
\]


\subsection*{Question 21, Chapter 2 (\cite{toi3rd})}

\textbf{21.}  

\bigskip
\textbf{(a)} Banker's Rule:  
It uses exact days (larger numerator) but divides by 360 instead of 365 (smaller denominator).  
Thus the fraction of the year is always larger than under exact simple interest.  
Therefore Banker's Rule always yields more interest for the lender.  

\[
\frac{t}{360} > \frac{t}{365}.
\]

\bigskip
\textbf{(b)} Compared to ordinary simple interest:  
Ordinary uses 30 days per month and 360-day year.  
Banker's Rule uses exact days (≥ ordinary days) with 360-day year.  
Thus Banker's Rule is usually more favorable to the lender.  

\bigskip
\textbf{(c)} Counterexample:  
Suppose investment spans February (28 days).  
Ordinary simple interest counts this as 30 days,  
while Banker's Rule counts 28 days.  
Thus ordinary simple interest gives higher interest in this case.  


\newpage
\printbibliography
\end{document}
