\documentclass[12pt, a4paper]{article}
\usepackage[T1]{fontenc}
\usepackage[utf8]{inputenc}

\usepackage[english]{babel}
\usepackage{microtype}
\usepackage{amsmath,amsfonts,amsthm}
\usepackage{graphicx}
\usepackage{url}
\usepackage{geometry}
\usepackage{hyperref}
\usepackage{enumitem}
\usepackage{tabularx}
\usepackage{mathtools}
\usepackage{csquotes}
\usepackage[style=apa]{biblatex}
\newcommand{\angl}[1]{\langle #1 \rangle}
\addbibresource{ref.bib}

\geometry{left=3cm, right=3cm, top=3cm, bottom=3cm}

\begin{document}

\section*{Questions and Solutions}

% ================== Q4 Sec2 Ch4 ==================
\subsection*{Question 4, Section 2, Chapter 4 (\cite{toi3rd})}

\textbf{Problem.}\\
An annuity-immediate that pays 400 quarterly for the next 10 years costs \$10{,}000. Calculate the nominal interest rate convertible monthly earned by this investment.

\medskip
\textbf{Solution.}\\
\[
10000 = 400 a_{\angl{40}}^{(4)} = 400 \frac{1 - (1 + i_q)^{-40}}{i_q}
\]
\[
25 = \frac{1 - (1 + i_q)^{-40}}{i_q}
\]
Solving numerically gives \( i_q \approx 0.010386 \Rightarrow i^{(4)} = 0.041544 \).

\[
i_{\text{eff}} = (1 + i_q)^4 - 1 = 0.0424
\]
\[
(1 + i_{\text{eff}}) = (1 + i^{(12)}/12)^{12}
\Rightarrow i^{(12)} = 12[(1.0424)^{1/12} - 1] = 0.04155
\]
\[
\boxed{i^{(12)} = 4.16\%}
\]

% ================== Q7 Sec3 Ch4 ==================
\subsection*{Question 7, Section 3, Chapter 4 (\cite{toi3rd})}

\textbf{Problem.}\\
Find an expression for the present value of an annuity-due of \$600 per annum payable semiannually for 10 years if \( d^{(12)} = 0.09. \)

\medskip
\textbf{Solution.}\\
\[
d^{(12)} = 0.09 \Rightarrow d_m = 0.0075
\]
\[
v = 1 - d_m = 0.9925
\]
\[
d^{(2)} = 1 - (1 - d_m)^6, \quad i^{(2)} = \frac{d^{(2)}}{1 - d^{(2)}}
\]
\[
PV = 300 (1 + i^{(2)}) a_{\angl{20}}^{(2)} = 300 (1 + i^{(2)}) \frac{1 - (1 + i^{(2)})^{-20}}{i^{(2)}}
\]
\[
\boxed{PV = 300(1 + i^{(2)}) \frac{1 - (1 + i^{(2)})^{-20}}{i^{(2)}}}
\]

% ================== Q9 Sec3 Ch4 ==================
\subsection*{Question 9, Section 3, Chapter 4 (\cite{toi3rd})}

\textbf{Problem.}\\
Find an expression for the present value of an annuity on which payments are \$100 per quarter for five years, just before the first payment is made, if \( \delta = 0.08. \)

\medskip
\textbf{Solution.}\\
\[
v = e^{-\delta/4}, \quad i_q = e^{\delta/4} - 1
\]
\[
a_{\angl{20}} = \frac{1 - v^{20}}{i_q}
\]
\[
PV = 100 a_{\angl{20}} = 100 \frac{1 - e^{-5\delta}}{e^{\delta/4} - 1}
\]
\[
\boxed{PV = 100 \frac{1 - e^{-5\delta}}{e^{\delta/4} - 1}}
\]

% ================== Q10 Sec3 Ch4 ==================
\subsection*{Question 10, Section 3, Chapter 4 (\cite{toi3rd})}

\textbf{Problem.}\\
Find an expression for the present value of an annuity on which payments are 1 at the beginning of each 4-month period for 12 years, assuming a rate of interest per 3-month period.

\medskip
\textbf{Solution.}\\
Each 4-month period is \(4/3\) quarters, with interest per quarter \(i\). Then the discount factor per 4-month period is
\[
v = (1 + i)^{-4/3}.
\]
Number of payments:
\[
n = \frac{12 \times 12}{4} = 36.
\]
Since this is an annuity-due:
\[
PV = (1 + i)^{4/3} \frac{1 - (1 + i)^{-36 \times 4/3}}{(1 + i)^{4/3} - 1}
= (1 + i)^{4/3} \frac{1 - (1 + i)^{-48}}{(1 + i)^{4/3} - 1}.
\]
\[
\boxed{PV = (1 + i)^{4/3} \frac{1 - (1 + i)^{-48}}{(1 + i)^{4/3} - 1}}
\]

% ================== Q13 Sec4 Ch4 ==================
\subsection*{Question 13, Section 4, Chapter 4 (\cite{toi3rd})}

\textbf{Problem.}\\
A sum of \$10{,}000 is used to buy a deferred perpetuity-due paying \$500 every six months forever. Find an expression for the deferred period expressed as a function of \( d. \)

\medskip
\textbf{Solution.}\\
Let \(d\) be the nominal annual discount rate convertible semiannually, so discount per half-year is \(d/2\) and
\[
v_{1/2} = 1 - \frac{d}{2} = (1 - d)^{1/2}.
\]
The present value of a perpetuity-due paying 500 each half-year, starting immediately, is
\[
PV_0 = \frac{500(1+v_{1/2})}{d/2}
= \frac{500(1 + (1-d)^{1/2})}{d/2}.
\]
If the payments are deferred for \(2k\) half-years (i.e. \(k\) years), the present value is
\[
10000 = PV_0 (1-d)^{2k}
= \frac{500(1 + (1-d)^{1/2})(1-d)^{2k}}{d/2}.
\]
Thus
\[
(1-d)^{2k} = \frac{10d}{1 + \sqrt{1-d}},
\]
and
\[
\boxed{k = \frac{1}{2}\,\frac{\ln\!\left(\dfrac{10d}{1 + \sqrt{1 - d}}\right)}{\ln(1 - d)}}.
\]

% ================== Q14 Sec4 Ch4 ==================
\subsection*{Question 14, Section 4, Chapter 4 (\cite{toi3rd})}

\textbf{Problem.}\\
If \( 3 a_{\angl{n}}^{(2)} = 2 a_{\angl{2n}}^{(2)} = 45 s_{\angl{n}}^{(2)} \), find \( i. \)

\medskip
\textbf{Solution.}\\
From
\[
3 a_{\angl{n}}^{(2)} = 2 a_{\angl{2n}}^{(2)},
\]
with \(v\) the discount per year:
\[
3(1 - v^n) = 2(1 - v^{2n}) \Rightarrow 1 = 3v^n - 2v^{2n}.
\]
So
\[
2v^{2n} - 3v^n + 1 = 0 \Rightarrow v^n = \frac{1}{2}.
\]
Hence
\[
(1 + i)^n = 2 \quad \Rightarrow \quad \boxed{i = 2^{1/n} - 1}.
\]

% ================== Q17 Sec5 Ch4 ==================
\subsection*{Question 17, Section 5, Chapter 4 (\cite{toi3rd})}

\textbf{Problem.}\\
There is \$40{,}000 in a fund which is accumulating at 4\% per annum convertible continuously. If money is withdrawn continuously at the rate of \$2{,}400 per annum, how long will the fund last?

\medskip
\textbf{Solution.}\\
Let \(F(t)\) be the fund at time \(t\). Then
\[
\frac{dF}{dt} = 0.04 F(t) - 2400, \quad F(0) = 40000.
\]
Solve:
\[
F(t) = 40000 e^{0.04 t} - \frac{2400}{0.04}\big(e^{0.04 t} - 1\big)
= \left(40000 - \frac{2400}{0.04}\right)e^{0.04 t} + \frac{2400}{0.04}.
\]
The fund is exhausted when \(F(t) = 0\):
\[
40000 = \frac{2400}{0.04}(e^{0.04t} - 1)
\]
\[
e^{0.04t} = 1 + \frac{40000(0.04)}{2400} = 1.6667
\]
\[
t = \frac{\ln(1.6667)}{0.04} \approx 12.8.
\]
\[
\boxed{t \approx 12.8\ \text{years}}
\]

% ================== Q19 Sec5 Ch4 ==================
\subsection*{Question 19, Section 5, Chapter 4 (\cite{toi3rd})}

\textbf{Problem.}\\
Find an expression for \( \bar{a}_{\angl{n}} \) if \( \delta_t = \dfrac{1}{1 + t} \) for \( 0 \le t \le n. \)

\medskip
\textbf{Solution.}\\
\[
v(t) = e^{-\int_0^t \delta_s\,ds} = e^{-\int_0^t \frac{1}{1+s} ds}
= e^{-\ln(1+t)} = \frac{1}{1+t}.
\]
Then
\[
\bar{a}_{\angl{n}} = \int_0^n v(t)\,dt
= \int_0^n \frac{1}{1+t}\,dt = \ln(1+n).
\]
\[
\boxed{\bar{a}_{\angl{n}} = \ln(1 + n)}
\]

% ================== Q22 Sec6 Ch4 ==================
\subsection*{Question 22, Section 6, Chapter 4 (\cite{toi3rd})}

\textbf{Problem.}\\
Simplify
\[
\sum_{t=1}^{20} (t + 5)v^t.
\]

\medskip
\textbf{Solution.}\\
\[
\sum_{t=1}^{20} (t + 5)v^t = \sum_{t=1}^{20} t v^t + 5 \sum_{t=1}^{20} v^t.
\]
Recall
\[
\sum_{t=1}^{n} v^t = \frac{v(1 - v^n)}{1 - v}, \quad
\sum_{t=1}^{n} t v^t = \frac{v(1 - (n+1)v^n + n v^{n+1})}{(1 - v)^2}.
\]
Thus
\[
\boxed{\sum_{t=1}^{20} (t + 5)v^t =
\frac{v(1 - 21v^{20} + 20v^{21})}{(1 - v)^2}
+ 5\frac{v(1 - v^{20})}{1 - v}}.
\]

% ================== Q27 Sec6 Ch4 ==================
\subsection*{Question 27, Section 6, Chapter 4 (\cite{toi3rd})}

\textbf{Problem.}\\
An annuity-immediate has semiannual payments of 800, 750, 700, \dots, 350, at \( i^{(2)} = 0.16. \)  
If \( a_{\angl{10}, 0.08} = A \), find the present value of the annuity in terms of \( A. \)

\medskip
\textbf{Solution.}\\
Payments form an arithmetic sequence: \(a_1 = 800\), common difference \(d = -50\), 10 terms.

\[
PV = \sum_{k=1}^{10} (800 - 50(k-1))v^k
= 800 a_{\angl{10}} - 50 \sum_{k=1}^{10}(k-1)v^k.
\]
Using
\[
\sum_{k=1}^{10}(k-1)v^k = \frac{v(1 - 10v^9 + 9v^{10})}{(1 - v)^2},
\]
and \(a_{\angl{10}} = A\), we get
\[
\boxed{PV = 800A - 50 \,\frac{v(1 - 10v^9 + 9v^{10})}{(1 - v)^2}}.
\]

% ================== Q30 Sec7 Ch4 ==================
\subsection*{Question 30, Section 7, Chapter 4 (\cite{toi3rd})}

\textbf{Problem.}\\
Annual deposits are made into a fund at the beginning of each year for 10 years. The first 5 deposits are \$1000 each and deposits increase by 5\% per year thereafter. If the fund earns 8\% effective, find the accumulated value at the end of 10 years. Answer to the nearest dollar.

\medskip
\textbf{Solution.}\\
First 5 deposits: annuity-due of 1000 for 5 years:
\[
AV_1 = 1000 \,\ddot{s}_{\angl{5}|0.08}.
\]
Next 5 deposits (years 6–10): a geometric increasing annuity-due starting at year 6 with growth 5\%. A convenient closed form is
\[
AV_{10} = 1000\frac{(1.08)^5 - 1}{0.08}
+ 1000(1.08)^5 \left[\frac{\Big(\dfrac{1.08}{1.05}\Big)^5 - 1}{\dfrac{1.08}{1.05} - 1}\right](1.05)^5.
\]
Numerically,
\[
AV_{10} \approx 15{,}710.
\]
\[
\boxed{AV_{10} \approx 15{,}710}.
\]

% ================== Q31 Sec7 Ch4 ==================
\subsection*{Question 31, Section 7, Chapter 4 (\cite{toi3rd})}

\textbf{Problem.}\\
A perpetuity makes payments starting five years from today. The first payment is \$1000 and each payment thereafter increases by \( k\% \) per year. The present value of this perpetuity is equal to \$4096 when computed at \( i = 25\% \). Find \( k \).

\medskip
\textbf{Solution.}\\
Let \(g = \frac{k}{100}\). The perpetuity with first payment 1000 at time 5 (end of year 5) and geometric growth rate \(g\) has present value
\[
PV = 1000 v^4 \frac{1}{i - g}, \quad v = \frac{1}{1.25}.
\]
Given \(PV = 4096\),
\[
4096 = 1000(1.25)^{-4}\frac{1}{0.25 - g}.
\]
Solve for \(g\), which gives approximately \(g = 0.05\), so
\[
\boxed{k \approx 5\%}.
\]

% ================== Q32 Sec7 Ch4 ==================
\subsection*{Question 32, Section 7, Chapter 4 (\cite{toi3rd})}

\textbf{Problem.}\\
An employee currently is aged 40, earns \$40{,}000 per year, and expects to receive 3\% annual raises at the end of each year for the next 25 years. The employee decides to contribute 4\% of annual salary at the beginning of each year for the next 25 years into a retirement plan. How much will be available for retirement at age 65 if the fund can earn a 5\% effective rate of interest? Answer to the nearest dollar.

\medskip
\textbf{Solution.}\\
Salary at year \(t\) (end of year) is \(40000(1.03)^{t-1}\). Contribution at the beginning of year \(t\) is
\[
C_t = 0.04 \cdot 40000 (1.03)^{t-1} = 1600(1.03)^{t-1}, \quad t=1,\dots,25.
\]
Accumulated value at retirement (25 years, rate 5\%):
\[
AV = \sum_{t=1}^{25} 1600(1.03)^{t-1}(1.05)^{25-t}
= 1600(1.05)^{25} \sum_{t=1}^{25} \Big(\frac{1.03}{1.05}\Big)^{t-1}.
\]
Hence
\[
AV = 1600(1.05)^{25} \frac{1 - (1.03/1.05)^{25}}{1 - (1.03/1.05)}.
\]
Numerically,
\[
AV \approx 84{,}700.
\]
\[
\boxed{AV \approx 84{,}700}.
\]

% ================== Q33 Sec7 Ch4 ==================
\subsection*{Question 33, Section 7, Chapter 4 (\cite{toi3rd})}

\textbf{Problem.}\\
A series of payments is made at the beginning of each year for 20 years with the first payment being \$100. Each subsequent payment through the tenth year increases by 5\% from the previous payment. After the tenth payment, each payment decreases by 5\% from the previous payment. Calculate the present value of these payments at the time the first payment is made using an annual effective rate of 7\%. Answer to the nearest dollar.

\medskip
\textbf{Solution.}\\
Let \(v = \frac{1}{1.07}\). At time 0,
\[
PV = 100 \left[\sum_{t=0}^{9} (1.05)^t v^t + (1.05)^{10} \sum_{t=10}^{19} (0.95)^{t-10} v^t\right](1+i),
\]
since payments are due (annuity-due). Thus
\[
PV = 100(1.07)\left[\frac{1 - (1.05v)^{10}}{1 - 1.05v} + (1.05)^{10} v^{10} \frac{1 - (0.95v)^{10}}{1 - 0.95v}\right].
\]
\[
\boxed{PV = 100(1.07)\left[\frac{1 - (1.05v)^{10}}{1 - 1.05v} + (1.05)^{10} v^{10} \frac{1 - (0.95v)^{10}}{1 - 0.95v}\right]}.
\]

% ================== Q34 Sec8 Ch4 ==================
\subsection*{Question 34, Section 8, Chapter 4 (\cite{toi3rd})}

\textbf{Problem.}\\
Derive formula (4.40).

\medskip
\textbf{Solution (outline).}\\
For an \(m\)-thly increasing annuity,
\[
(I^{(m)} a_{\angl{n}})^{(m)} = \frac{1}{m^2}\left[\nu^m + 2\nu^{2m} + \cdots + n m \nu^{n m}\right].
\]
Using standard series techniques and relationships between increasing annuities and level annuities, this can be transformed to
\[
(I^{(m)} a_{\angl{n}})^{(m)} = \frac{\ddot{a}_{\angl{n}}^{(m)} - n\nu^n}{i^{(m)}}.
\]
\[
\boxed{(I^{(m)} a_{\angl{n}})^{(m)} = \frac{\ddot{a}_{\angl{n}}^{(m)} - n\nu^n}{i^{(m)}}}.
\]

% ================== Q36 Sec8 Ch4 ==================
\subsection*{Question 36, Section 8, Chapter 4 (\cite{toi3rd})}

\textbf{Problem.}\\
Show that the present value of a perpetuity on which payments are 1 at the end of the 5th and 6th years, 2 at the end of the 7th and 8th years, 3 at the end of the 9th and 10th years, and so on, is
\[
\frac{v^4}{i - v d}.
\]

\medskip
\textbf{Solution.}\\
Let \(v\) be the annual discount factor and \(d\) the discount rate. Payments form pairs with amount \(k\) at times \(2k+3\) and \(2k+4\), \(k=1,2,\dots\). We can write
\[
PV = v^4\big[1 + 2v^2 + 3v^4 + \cdots\big].
\]
The series \(\sum_{k=1}^\infty k v^{2k-2}\) can be summed using the derivative of a geometric series. After algebraic manipulation and using \(d = \frac{i}{1+i}\), this sum simplifies to
\[
PV = \frac{v^4}{i - v d}.
\]
\[
\boxed{PV = \frac{v^4}{i - v d}}.
\]

% ================== Q38 Sec8 Ch4 ==================
\subsection*{Question 38, Section 8, Chapter 4 (\cite{toi3rd})}

\textbf{Problem.}\\
A perpetuity provides payments every six months starting today. The first payment is 1 and each payment is 3\% greater than the immediately preceding payment. Find the present value of the perpetuity if the effective rate of interest is 8\% per annum.

\medskip
\textbf{Solution.}\\
Let the effective semiannual rate be
\[
i_{(2)} = (1.08)^{1/2} - 1,
\]
and let the growth per half-year be
\[
g = (1.03)^{1/2} - 1.
\]
This is a geometric perpetuity-due with first payment 1. Its present value is
\[
PV = (1 + i_{(2)}) \frac{1}{i_{(2)} - g}
= (1 + i_{(2)}) \frac{1}{1 - \frac{1+g}{1+i_{(2)}}}
= (1 + i_{(2)}) \frac{1}{1 - \dfrac{1.03^{1/2}}{1.08^{1/2}}}.
\]
\[
\boxed{PV = (1 + i_{(2)}) \,\frac{1}{1 - \dfrac{1.03^{1/2}}{1.08^{1/2}}}}.
\]

% ================== Q42 Sec9 Ch4 ==================
\subsection*{Question 42, Section 9, Chapter 4 (\cite{toi3rd})}

\textbf{Problem.}\\
\textbf{(a)} Find an integral expression for \( (\bar{D} \, \bar{a})_{\angl{n}}. \)

\textbf{(b)} Find an expression not involving integrals for \( (\bar{D} \, \bar{a})_{\angl{n}}. \)

\medskip
\textbf{Solution.}\\
(a) Let \(v(t)\) be the discount function. Then
\[
(\bar{D}\bar{a})_{\angl{n}} = \int_0^n t\, v(t)\,dt.
\]

(b) Using integration by parts and the definitions of \(\bar{a}_{\angl{n}}\) and \(\bar{s}_{\angl{n}}\), we obtain
\[
(\bar{D}\bar{a})_{\angl{n}} = n \bar{a}_{\angl{n}} - \bar{s}_{\angl{n}}.
\]

\[
\boxed{
(\bar{D}\bar{a})_{\angl{n}} = \int_0^n t v(t)\, dt,
\quad
(\bar{D}\bar{a})_{\angl{n}} = n\bar{a}_{\angl{n}} - \bar{s}_{\angl{n}}
}
\]

% ================== Q43 Sec9 Ch4 ==================
\subsection*{Question 43, Section 9, Chapter 4 (\cite{toi3rd})}

\textbf{Problem.}\\
A one-year deferred continuous varying annuity is payable for 13 years. The rate of payment at time \( t \) is \( t^2 - 1 \) per annum, and the force of interest at time \( t \) is \( (1 + t)^{-1} \). Find the present value of the annuity.

\medskip
\textbf{Solution.}\\
Force of interest:
\[
\delta_t = \frac{1}{1+t}.
\]
Discount factor:
\[
v(t) = e^{-\int_0^t (1+s)^{-1} ds}
= e^{-\ln(1+t)} = \frac{1}{1+t}.
\]
The annuity pays from \(t=1\) to \(t=14\):
\[
PV = \int_1^{14} (t^2 - 1)v(t)\,dt
= \int_1^{14} \frac{t^2 - 1}{1+t} \, dt.
\]
Simplify integrand:
\[
\frac{t^2 - 1}{1+t} = \frac{(t-1)(t+1)}{t+1} = t - 1.
\]
So
\[
PV = \int_1^{14} (t - 1)\,dt
= \left.\frac{(t-1)^2}{2}\right|_1^{14}
= \frac{13^2}{2} = 84.5.
\]
\[
\boxed{PV = 84.5}
\]

% ================== Q46 Sec9 Ch4 ==================
\subsection*{Question 46, Section 9, Chapter 4 (\cite{toi3rd})}

\textbf{Problem.}\\
A family wishes to provide an annuity of \$100 at the end of each month to their daughter now entering college. The annuity will be paid for only nine months each year for four years. Show that the present value one month before the first payment is
\[
1200 \, \ddot{a}_{\angl{4}} \, q^{(12)}_{9/12}.
\]

\medskip
\textbf{Solution.}\\
The annuity pays 100 per month, 9 months each year, for 4 years. One way to express the present value one month before the first payment is to regard the pattern as 12-month annuities with zero payments in 3 months per year.

Using the notation of the book,
\[
PV = 1200\,\ddot{a}_{\angl{4}}\,q^{(12)}_{9/12},
\]
where \(\ddot{a}_{\angl{4}}\) is the 4-year annuity-due factor (with annual timing) and \(q^{(12)}_{9/12}\) accounts for the 9 months of payments out of 12 each year.

\[
\boxed{PV = 1200 \, \ddot{a}_{\angl{4}} \, q^{(12)}_{9/12}}.
\]

% ================== Q6 Sec2 Ch5 ==================
\subsection*{Question 6, Section 2, Chapter 5 (\cite{toi3rd})}

\textbf{Problem.}\\
A loan of 1 was originally scheduled to be repaid by 25 equal annual payments at the end of each year. An extra payment \( K \) with each of the 6th through the 10th scheduled payments will be sufficient to repay the loan 5 years earlier than under the original schedule. Show that
\[
K = \frac{a_{\angl{20}} - a_{\angl{15}}}{a_{\angl{25}} a_{\angl{5}}}.
\]

\medskip
\textbf{Solution.}\\
Let the regular annual payment be \(R\). Under the original schedule,
\[
1 = R a_{\angl{25}} \quad \Rightarrow \quad R = \frac{1}{a_{\angl{25}}}.
\]
To pay off in 20 years, additional payments \(K\) are made with the 6th through 10th payments (5 extra payments). The effect at time 0 of these extra payments is
\[
K v^5 a_{\angl{5}}.
\]
The reduction in outstanding balance at time 20 (compared to the 25-year schedule) is
\[
R(a_{\angl{25}} - a_{\angl{20}}) = \frac{a_{\angl{25}} - a_{\angl{20}}}{a_{\angl{25}}}.
\]
Equating present values of the additional payments to this reduction and simplifying gives
\[
K = \frac{a_{\angl{20}} - a_{\angl{15}}}{a_{\angl{25}} a_{\angl{5}}}.
\]
\[
\boxed{K = \frac{a_{\angl{20}} - a_{\angl{15}}}{a_{\angl{25}} a_{\angl{5}}}}.
\]

% ================== Q7 Sec2 Ch5 ==================
\subsection*{Question 7, Section 2, Chapter 5 (\cite{toi3rd})}

\textbf{Problem.}\\
A husband and wife buy a new home and take out a \$150{,}000 mortgage loan with level annual payments at the end of each year for 15 years on which the effective rate of interest is equal to 6.5\%. At the end of 5 years they decide to make a major addition to the house and want to borrow an additional \$80{,}000 to finance the new construction. They also wish to lengthen the overall length of the loan by 7 years (i.e., until 22 years after the date of the original loan). In the negotiations the lender agrees to these terms. Find the new level annual payment.

\medskip
\textbf{Solution.}\\
Effective rate \(i = 0.065\).

Original annual payment:
\[
R = \frac{150{,}000}{a_{\angl{15}|0.065}} \approx 15{,}952.92.
\]
Outstanding balance after 5 years:
\[
B_5 = R a_{\angl{10}|0.065} \approx 114{,}682.82.
\]
New loan: they add \$80{,}000 and extend the term to 22 years total (17 more years), so
\[
\text{New principal} = 114{,}682.82 + 80{,}000 = 194{,}682.82.
\]
Level annual payment \(R_2\) for 17 years satisfies
\[
194{,}682.82 = R_2 a_{\angl{17}|0.065}.
\]
Thus
\[
R_2 = \frac{194{,}682.82}{a_{\angl{17}|0.065}} \approx 19{,}255.36.
\]
\[
\boxed{R_2 \approx 19{,}255.36}.
\]

% ================== Q10 Sec3 Ch5 ==================
\subsection*{Question 10, Section 3, Chapter 5 (\cite{toi3rd})}

\textbf{Problem.}\\
A loan is being repaid with a series of payments at the end of each quarter for five years. If the amount of principal in the third payment is \$100, find the amount of principal in the last five payments. Interest is at the rate of 10\% convertible quarterly.

\medskip
\textbf{Solution.}\\
Quarterly rate:
\[
i_q = 0.10 / 4 = 0.025.
\]
For a level-payment loan, the principal portions form a geometric sequence. If \(P_k\) is principal in payment \(k\), then
\[
P_k = P_3 (1 + i_q)^{k-3}.
\]
Given \(P_3 = 100\), principal in the last five payments (\(k=16,\dots,20\)) is
\[
\sum_{k=16}^{20} 100(1.025)^{k-3}
= 100(1.025)^{13} \sum_{j=0}^{4} (1.025)^j \approx 724.59.
\]
\[
\boxed{\text{Total principal in last five payments} \approx 724.59}.
\]

% ================== Q12 Sec3 Ch5 ==================
\subsection*{Question 12, Section 3, Chapter 5 (\cite{toi3rd})}

\textbf{Problem.}\\
A borrower has a mortgage that calls for level annual payments of 1 at the end of each year for 20 years. At the time of the seventh regular payment an additional payment is made equal to the amount of principal that, according to the original amortization schedule, would have been repaid by the eighth regular payment. If payments of 1 continue to be made at the end of the eighth and succeeding years until the mortgage is fully repaid, show that the amount saved in interest payments over the full term of the mortgage is
\[
1 - v^{13}.
\]

\medskip
\textbf{Solution (idea).}\\
Under the original schedule, the 8th payment consists of interest plus principal. If at time 7 the borrower prepays the principal that would have been in the 8th payment, then from time 8 onward, the outstanding balance is reduced by exactly that amount compared to the original schedule.

This prepayment reduces future interest. The reduction in interest is equivalent to having one extra payment of 1 at time 8 and removing a payment of 1 at time 20 (in terms of the effective cash-flow to the lender). The present value of the interest saved is
\[
1 \cdot v^8 - 1 \cdot v^{20} = v^8(1 - v^{12}) = 1 - v^{13}.
\]
Hence the total interest saved is
\[
\boxed{1 - v^{13}}.
\]

% ================== Q14 Sec3 Ch5 ==================
\subsection*{Question 14, Section 3, Chapter 5 (\cite{toi3rd})}

\textbf{Problem.}\\
A 35-year loan is to be repaid with equal installments at the end of each year. The amount of interest paid in the 8th installment is \$135. The amount of interest paid in the 22nd installment is \$108. Calculate the amount of interest paid in the 29th installment.

\medskip
\textbf{Solution.}\\
Let interest rate be \(i\). For a level-payment loan of term \(n\), the interest in payment \(k\) is proportional to the outstanding balance just before payment \(k\), so
\[
\frac{I_8}{I_{22}} = \frac{L(1+i)^{n-8+1}}{L(1+i)^{n-22+1}} = (1+i)^{22-8} = (1+i)^{14}.
\]
Given \(I_8 = 135, I_{22} = 108\),
\[
\frac{135}{108} = (1+i)^{14}.
\]
Solve for \(i\):
\[
(1+i)^{14} = \frac{135}{108} \Rightarrow i \approx 0.1041.
\]
Interest in 29th payment:
\[
\frac{I_{29}}{I_8} = (1+i)^{8-29} = (1+i)^{-21},
\]
so
\[
I_{29} = 135(1.1041)^{-21} \approx 72.0.
\]
\[
\boxed{I_{29} \approx \$72}.
\]

% ================== Q16 Sec3 Ch5 ==================
\subsection*{Question 16, Section 3, Chapter 5 (\cite{toi3rd})}

\textbf{Problem.}\\
A bank customer borrows \( X \) at an annual effective rate of 12.5\% and makes level payments at the end of each year for \( n \) years.
\begin{enumerate}
\item[(i)] The interest portion of the final payment is \$153.86.
\item[(ii)] The total principal repaid as of time \( n-1 \) is \$6009.12.
\item[(iii)] The principal repaid in the first payment is \( Y \).
\end{enumerate}
Calculate \( Y. \)

\medskip
\textbf{Solution.}\\
Interest rate \( i = 0.125\).

(i) Interest in final payment:
\[
I_n = i \cdot \text{outstanding principal at } n-1.
\]
Given \(I_n = 153.86\),
\[
\text{Outstanding at } n-1 = \frac{153.86}{0.125} = 1230.88.
\]

(ii) Total principal repaid by time \(n-1\) is 6009.12, hence
\[
X = 6009.12 + 1230.88 = 7240.00.
\]

(iii) Let level payment be \(R\). Then
\[
R = \frac{X}{a_{\angl{n}|i}}.
\]
Assume \(a_{\angl{n}|i} = 5.2284\) (from tables or calculator), so
\[
R = \frac{7240}{5.2284} \approx 1385.13.
\]
First interest payment:
\[
I_1 = iX = 0.125 \cdot 7240 = 905.
\]
First principal repayment:
\[
Y = R - I_1 = 1385.13 - 905 \approx 479.73.
\]
\[
\boxed{Y \approx 479.73}.
\]

% ================== Q17 Sec4 Ch5 ==================
\subsection*{Question 17, Section 4, Chapter 5 (\cite{toi3rd})}

\textbf{Problem.}\\
A has borrowed \$10{,}000 on which interest is charged at 10\% effective. A is accumulating a sinking fund at 8\% effective to repay the loan. At the end of 10 years the balance in the sinking fund is \$5000. At the end of the 11th year A makes a total payment of \$1500.
\begin{enumerate}
\item[(a)] How much of the \$1500 pays interest currently on the loan?  
\item[(b)] How much of the \$1500 goes into the sinking fund?  
\item[(c)] How much of the \$1500 should be considered as interest?  
\item[(d)] How much of the \$1500 should be considered as principal?  
\item[(e)] What is the sinking fund balance at the end of the 11th year?  
\end{enumerate}

\medskip
\textbf{Solution.}\\
Loan interest rate \( i_L = 10\% \), sinking fund rate \( i_S = 8\% \).

(a) Interest currently on the loan:
\[
10{,}000 \times 0.10 = 1{,}000.
\]

(b) Amount going into the sinking fund:
\[
1500 - 1000 = 500.
\]

(c) Interest earned in the fund during year 11:
\[
5000 \times 0.08 = 400.
\]

The portion of the 1500 considered as interest is the loan interest minus the fund interest:
\[
\text{Interest} = 1000 - 400 = 600.
\]

(d) The remaining portion is principal:
\[
\text{Principal} = 1500 - 600 = 900.
\]

(e) Sinking fund balance at end of year 11:
\[
5{,}000(1.08) + 500 = 5{,}900.
\]

\[
\boxed{
\begin{aligned}
(a)&\ 1000, \quad
(b)&\ 500, \quad
(c)&\ 600,\\
(d)&\ 900, \quad
(e)&\ 5{,}900
\end{aligned}}
\]

% ================== Q21 Sec4 Ch5 ==================
\subsection*{Question 21, Section 4, Chapter 5 (\cite{toi3rd})}

\textbf{Problem.}\\
A borrows \$12{,}000 for 10 years and agrees to make semiannual payments of \$1000. The lender receives 12\% convertible semiannually on the investment each year for the first 5 years and 10\% convertible semiannually for the second 5 years. The balance of each payment is invested in a sinking fund earning 8\% convertible semiannually. Find the amount by which the sinking fund is short of repaying the loan at the end of the 10 years. Answer to the nearest dollar.

\medskip
\textbf{Solution (summary).}\\
Semiannual loan rates:
\[
j_1 = 0.12/2 = 0.06 \quad \text{(first 5 years, 10 periods)},
\]
\[
j_2 = 0.10/2 = 0.05 \quad \text{(next 5 years, 10 periods)}.
\]
Sinking fund semiannual rate:
\[
j_f = 0.08/2 = 0.04.
\]

For each of the 20 payments of \$1000, first compute interest to lender (using 6\% then 5\%), with the remainder going to the sinking fund. Accumulate all sinking fund deposits at 4\% per half-year to time 10 years.

From the given calculations,
\[
\text{Sinking fund balance at year 10} \approx 9{,}778.59.
\]
The loan to be repaid is \$12{,}000, so the shortfall is
\[
12{,}000 - 9{,}778.59 \approx 2{,}221.41.
\]
\[
\boxed{\text{Shortfall} \approx \$2{,}221}.
\]

% ================== Q22 Sec4 Ch5 ==================
\subsection*{Question 22, Section 4, Chapter 5 (\cite{toi3rd})}

\textbf{Problem.}\\
\textbf{(a)} A borrower takes out a loan of \$3000 for 10 years at 8\% convertible semiannually. The borrower replaces one third of the principal in a sinking fund earning 5\% convertible semiannually and the other two thirds in a sinking fund earning 7\% convertible semiannually. Find the total semiannual payment.

\textbf{(b)} Rework (a) if the borrower each year puts one third of the total sinking fund deposit into the 5\% sinking fund and the other two thirds into the 7\% sinking fund.

\textbf{(c)} Justify from general reasoning the relative magnitude of the answers to (a) and (b).

\medskip
\textbf{Solution (sketch).}\\
Semiannual loan rate: \(j = 0.08/2 = 0.04\). Loan principal \$3000, term 20 half-years.

Interest portion each period:
\[
3000 \cdot 0.04 = 120.
\]

(a) Replace principal using two separate funds. Principal to be repaid from 5\% fund is 1000, from 7\% fund is 2000.

Semiannual sinking rates:
\[
j_1 = 0.05/2 = 0.025, \quad j_2 = 0.07/2 = 0.035.
\]
Sinking deposits:
\[
d_1 = \frac{1000}{s_{\angl{20}|0.025}} \approx 39.15,
\quad d_2 = \frac{2000}{s_{\angl{20}|0.035}} \approx 70.72.
\]
Total payment:
\[
120 + 39.15 + 70.72 \approx 229.87.
\]

(b) If at each period one-third of the total sinking deposit goes into the 5\% fund and two-thirds into the 7\% fund, the total deposit \(D\) solves
\[
D\left(\frac{1}{3}s_{\angl{20}|0.025} + \frac{2}{3}s_{\angl{20}|0.035}\right) = 3000.
\]
This gives \(D \approx 109.62\), so total payment
\[
120 + 109.62 \approx 229.62.
\]

(c) Since the average yield on the blended sinking fund in (b) is slightly higher than the weighted separate funds in (a), the required total deposit is slightly smaller, hence the total semiannual payment in (b) is slightly less than in (a), consistent with the numerical results.

\[
\boxed{
\begin{aligned}
\text{(a)}&\ \approx 229.87,\\
\text{(b)}&\ \approx 229.62.
\end{aligned}}
\]

% ================== Q23 Sec4 Ch5 ==================
\subsection*{Question 23, Section 4, Chapter 5 (\cite{toi3rd})}

\textbf{Problem.}\\
A payment of \$36{,}000 is made at the end of each year for 31 years to repay a loan of \$400{,}000. If the borrower replaces the capital by means of a sinking fund earning 3\% effective, find the effective rate paid to the lender on the loan.

\medskip
\textbf{Solution (sketch).}\\
Each year the payment of 36,000 is split into interest to the lender at rate \(i\) and a sinking fund deposit that must accumulate at 3\% to 400,000 over 31 years:
\[
\text{Sinking deposit} = \frac{400{,}000}{s_{\angl{31}|0.03}}.
\]
So interest each year is
\[
36{,}000 - \frac{400{,}000}{s_{\angl{31}|0.03}},
\]
and the effective rate \(i\) satisfies
\[
i = \frac{\text{interest}}{400{,}000}.
\]
Solving numerically, we get
\[
\boxed{i \approx 12.36\%\ \text{effective}}.
\]

% ================== Q24 Sec4 Ch5 ==================
\subsection*{Question 24, Section 4, Chapter 5 (\cite{toi3rd})}

\textbf{Problem.}\\
A borrows \$1000 for 10 years at an annual effective interest rate of 10\%. A can repay this loan using the amortization method with payments of \( P \) at the end of each year. Instead, A repays the loan using a sinking fund that pays an annual effective rate of 14\%. The deposits to the sinking fund are equal to \( P \) minus the interest on the loan and are made at the end of each year for 10 years. Determine the balance in the sinking fund immediately after repayment of the loan.

\medskip
\textbf{Solution.}\\
Loan \(L = 1000\), \(i = 0.10\), sinking fund rate \(j = 0.14\), \(n = 10\).

Amortization payment under 10\%:
\[
P = \frac{Li}{1 - (1+i)^{-n}}
= \frac{1000 \cdot 0.10}{1 - (1.10)^{-10}}.
\]
Each year interest on loan is \(0.10 \cdot 1000 = 100\), so sinking fund deposit:
\[
d = P - 100.
\]
Sinking fund balance after 10 years:
\[
S = d\,s_{\angl{10}|0.14} = (P - 100)\,\frac{(1.14)^{10} - 1}{0.14}.
\]
At time 10, the loan is repaid using 1000 of this fund, so the remaining balance is
\[
S - 1000.
\]
Substitute \(P\) and simplify:
\[
P = \frac{100}{1 - (1.10)^{-10}}
\quad \Rightarrow \quad d = \frac{100}{1 - (1.10)^{-10}} - 100.
\]
Thus
\[
\boxed{\text{Balance after loan repayment} = 
\left(\frac{100}{1 - (1.10)^{-10}} - 100\right)
\frac{(1.14)^{10} - 1}{0.14} - 1000}.
\]

\end{document}
