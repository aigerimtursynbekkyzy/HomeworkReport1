\documentclass[12pt, a4paper]{article}
\usepackage[T1]{fontenc}

\usepackage[english]{babel}
\usepackage{microtype}
\usepackage{amsmath,amsfonts,amsthm}
\usepackage{graphicx}
\usepackage{url}
\usepackage{geometry}
\usepackage{hyperref}
\usepackage{fancyhdr}
\usepackage{enumitem}
\usepackage{tabularx}
\usepackage{mathtools}
\usepackage{csquotes}
\usepackage[style=apa]{biblatex}
\addbibresource{ref.bib}
% Adjust margins here

\geometry{left=3cm, right=3cm, top=3cm, bottom=3cm, headheight=15pt}
\addtolength{\topmargin}{-2.5pt}
% Increase bottom margin to lower page numbering

\pagestyle{fancy}
\fancyhf{} % clear all header and footer fields
\fancyhead[L]{MATH 323: Actuarial Mathematics I  } % left header
\fancyhead[R]{Homework Report #1} % right header
\fancyfoot[C]{\thepage} % center footer
\renewcommand{\headrulewidth}{0.4pt} % header rule width
\renewcommand{\footrulewidth}{0.4pt} % footer rule width

\begin{document}

\begin{titlepage}
    \centering
    
    \vspace*{0.5cm}
    
    {\Large\bfseries MATH 323: Actuarial Mathematics I\par}
    
    \vspace{1cm}
    
    {\large Homework Report #1\par}
    
    \vspace{0.5cm}
    
    {\today\par}
    
    \vspace{1pt}
    
    \includegraphics[width=0.3\textwidth]{NU-logo.png}\\
    \includegraphics[width=0.15\textwidth]{sosah-logo.png}

    \vspace{0.5cm}
    
    Submitted for {\bf MATH 323: Actuarial Mathematics I} at the School of Sciences and Humanities, Department of Mathematics, Nazarbayev University
    
    \vspace{0.5cm}
    
    {\large Student Name:\par}
    \begin{itemize}[leftmargin=5cm,rightmargin=4cm]
        \item  Aigerim Tursynbekkyzy - ID: 202043550
    \end{itemize}

    \vspace{0.5cm}
    
    \flushleft{
  Subject Area: {\bf Theory of Interest} \\
  Description: {\bf Homework Problems in Chapter 1 and Chapter 22 \\
  Course Instructor : {\bf Dongming Wei} \\
    }
    
    \vspace{0.5cm}
    
    {\footnotesize In submitting this work we are indicating
    that we have read the University's Academic Integrity Policy. We
    declare that all material in this assessment is our own work except
    where there is clear acknowledgment and reference to the work of
    others.\par}
\end{titlepage}
The following is the used as solutions samples for each problem:
\newpage
\section*{Problems}
\subsection*{Question 19, Chapter 1  (\cite{toi3rd})}
\noindent It is known that an investment of \$500 will increase to \$4000 at the end of 30 years. Find the sum of the present values of three payments of \$10,000 each which will occur at the end of 20, 40, and 60 years.

\subsection*{Question 24, Chapter 1  (\cite{toi3rd})}
\noindent Show that
\[
\frac{d^3}{(1 - d)^2} = \frac{(i - d)^2}{1 - v}.
\]

\subsection*{Question 26, Chapter 1  (\cite{toi3rd})}
\noindent
\begin{enumerate}
    \item[(a)] Express \( d^{(4)} \) as a function of \( i^{(3)} \).
    \item[(b)] Express \( i^{(6)} \) as a function of \( d^{(2)} \).
\end{enumerate}

\subsection*{Question 27, Chapter 1  (\cite{toi3rd})}
\noindent
\begin{enumerate}
    \item[(a)] Show that \( i^{(m)} = d^{(m)} (1 + i)^{1/m} \).
    \item[(b)] Verbally interpret the result obtained in (a).
\end{enumerate}

\subsection*{Question 28, Chapter 1  (\cite{toi3rd})}
\noindent Find the accumulated value of \$100 at the end of two years:
\begin{enumerate}
    \item[(a)] If the nominal annual rate of interest is 6\% convertible quarterly.
    \item[(b)] If the nominal annual rate of discount is 6\% convertible once every four years.
\end{enumerate}

\subsection*{Question 34, Chapter 1  (\cite{toi3rd})}
\noindent Fund A accumulates at a simple interest rate of 10\%. Fund B accumulates at a simple discount rate of 5\%. Find the point in time at which the forces of interest on the two funds are equal.

\subsection*{Question 35, Chapter 1  (\cite{toi3rd})}
\noindent An investment is made for one year in a fund whose accumulation function is a second degree polynomial. The nominal rate of interest earned during the first half of the year is 5\% convertible semiannually. The effective rate of interest earned for the entire year is 7\%. Find \( \delta_s \).

\subsection*{Question 37, Chapter 1  (\cite{toi3rd})}
\noindent Find the level effective rate of interest over a three-year period which is equivalent to an effective rate of discount of 8\% the first year, 7\% the second year, and 6\% the third year.

\subsection*{Question 46, Chapter 1  (\cite{toi3rd})}
\noindent You are given \( \delta_t = \frac{2}{t-1} \) for \( 2 \leq t \leq 10 \). For any one-year interval between times \( n \) and \( n + 1 \), with \( n = 2, 3, \dots, 9 \), calculate the equivalent \( d^{(2)} \).

\subsection*{Question 49, Chapter 1  (\cite{toi3rd})}
\noindent Find the following derivatives:
\begin{enumerate}
    \item[(a)] \( \frac{dd}{di} \)
    \item[(b)] \( \frac{d\delta}{di} \)
    \item[(c)] \( \frac{d\delta}{dv} \)
    \item[(d)] \( \frac{dd}{d\delta} \)
\end{enumerate}


\subsection*{Question 2, Chapter 2 (\cite{toi3rd})}

\noindent You have an inactive credit card with a \$1000 outstanding unpaid balance. This particular credit card charges interest at the rate of 18\% compounded monthly. You are able to make a payment of \$200 one month from today and \$300 two months from today. Find the amount that you will have to pay three months from today to completely pay off this credit card debt. (Note: Work this problem with an equation of value. You will learn an alternative approach for this type of problem in Chapter 5.)

\bigskip


\subsection*{Question 3, Chapter 2 (\cite{toi3rd})}

\noindent At a certain interest rate the present value of the following two payment patterns are equal:

\[
(i) \quad 200 \text{ at the end of 5 years plus } 500 \text{ at the end of 10 years.}
\]

\[
(ii) \quad 400.94 \text{ at the end of 5 years.}
\]

At the same interest rate, \$100 invested now plus \$120 invested at the end of 5 years will accumulate to \( P \) at the end of 10 years. Calculate \( P \).

\bigskip


\subsection*{Question 5, Chapter 2 (\cite{toi3rd})}

\noindent Whereas the choice of a comparison date has no effect on the answer obtained with compound interest, the same cannot be said of simple interest. Find the amount to be paid at the end of 10 years which is equivalent to two payments of \$100 each, the first to be paid immediately and the second to be paid at the end of 5 years. Assume 5\% simple interest is earned from the date each payment is made and use a comparison date of:

\[
\text{a) The end of 10 years.}
\]
\[
\text{b) The end of 15 years.}
\]

\bigskip



\subsection*{Question 10, Chapter 2 (\cite{toi3rd})}

\noindent You are asked to develop a rule of \( n \) to approximate how long it takes money to triple. Find \( n \), where \( n \) is a positive integer.

\bigskip



\subsection*{Question 11, Chapter 2 (\cite{toi3rd})}

\noindent A deposits 10 today and another 30 in five years into a fund paying simple interest of 11\% per year. B will make the same two deposits, but the 10 will be deposited \( n \) years from today and the 30 will be deposited \( 2n \) years from today. B’s deposits earn an annual effective rate of 9.15\%. At the end of 10 years, the accumulated value of B’s deposits equals the accumulated value of A’s deposits. Calculate \( n \).

\bigskip


\subsection*{Question 13, Chapter 2 (\cite{toi3rd})}

\noindent Find the nominal rate of interest convertible semiannually at which the accumulated value of \$1000 at the end of 15 years is \$3000.

\bigskip



\subsection*{Question 16, Chapter 2 (\cite{toi3rd})}

\noindent It is known that an investment of \$1000 will accumulate to \$1825 at the end of 10 years. If it is assumed that the investment earns simple interest at rate \( i \) during the 1st year, \( 2i \) during the 2nd year, ..., \( 10i \) during the 10th year, find \( i \).

\bigskip



\subsection*{Question 17, Chapter 2 (\cite{toi3rd})}

\noindent It is known that an amount of money will double itself in 10 years at a varying force of interest \( \delta_t = kt \). Find an expression for \( k \).

\bigskip



\subsection*{Question 18, Chapter 2 (\cite{toi3rd})}

\noindent The sum of the accumulated value of 1 at the end of three years at a certain effective rate of interest \( i \), and the present value of 1 to be paid at the end of three years at an effective rate of discount numerically equal to \( i \) is 2.0096. Find the rate \( i \).

\bigskip


\subsection*{Question 20, Chapter 2 (\cite{toi3rd})}

\noindent A sum of \$10,000 is invested for the months of July and August at 6\% simple interest. Find the amount of interest earned:

\[
\text{a) Assuming exact simple interest.}
\]
\[
\text{b) Assuming ordinary simple interest.}
\]
\[
\text{c) Assuming the Banker’s Rule.}
\]

\bigskip


\subsection*{Question 21, Chapter 2 (\cite{toi3rd})}
\[
\text{a) Show that the Banker’s Rule is always more favorable to the lender than is exact simple interest.}
\]
\[
\text{b) Show that the Banker’s Rule is usually more favorable to the lender than is ordinary simple interest.}
\]
\[
\text{c) Find a counterexample in (b) for which the opposite relationship holds.}
\]

\newpage

\section*{Solutions}

\subsection*{Question 1, Section 2, Chapter 2  (\cite{toi3rd})}

\noindent\textbf{Assigned to Dina Kalibekova}\\

\noindent The number of ways that five cards are dealt from a standard 52-card deck is:
$$
    C=\frac{52!}{5!(52-5)!}= 2598960
$$
4 cards which contain the number 10 and one card which contains 9:
$$
    (1\times4)+(4\times6)+(4\times1)=32
$$
The probability that the sum of the faces on the five cards is 48 or more is:
$$
   P(Sum \ \ is \ \ 48 \ \ or \ \ more) = \frac{m}{n} = \frac{32}{2598960},$$
\newline
where m - The number of ways of selecting the sum of the faces and n - Total number of ways
\newline
Therefore, the probability that the sum of the faces on the five cards is 48 or more is:

    $$\frac{32}{2598960}$$

\subsection*{Question 2, Section 2, Chapter 2 (\cite{toi3rd})}

\noindent\textbf{Assigned to Zhannur Kazenov}\\

\noindent To get a flush, Dana needs to draw any 3 of the remaining 11 diamonds. Since only 47 cards are effectively left in the deck (others may already have been dealt, but their identities are unknown), 
\newline P(Dana draws to flush) = P(A)
\[
    P(A) =\frac{\binom{11}{3}}{\binom{47}{3}}=0.0101758
\]
\subsection*{Question 3, Section 2, Chapter 2 (\cite{toi3rd})}

\noindent\textbf{Assigned to Alina Abdrakhmanova}\\

\noindent Tim is dealt a 4 of a clubs, 8 of hearts, 9 of hearts, and king of diamonds. So, he has been dealt 5 cards and there are 47 cards remaining in the deck. To get a straight flush, Tim needs to have all the cards of the same suit and in sequence. He has 3 hearts in hand that can be made into sequence if 7 of hearts is added to it. Then another 1 card is needed (either 5 or 10 of hearts). \newline
7 of hearts can be drawn in \begin{equation}
    \binom{1}{1}
\end{equation} ways. \newline
Any of 5 or 10 of hearts can be drawn in \begin{equation}
    \binom{2}{1}
\end{equation} ways. \newline
Let A be an event that a straight flush is drawn. \newline
Then probability of it happening is: \begin{equation}
    P(A) =\frac{\binom{1}{1}\binom{2}{1}}{\binom{47}{2}}
\end{equation} \newline
To get a flush, which is when all cards are of the same suit and not necessarily in sequence, Tim needs to choose 2 more cards from of hearts. He can do it in $\binom{10}{2}$ ways. It includes 2 cards that will make a sequence a \textbf{straight flush}. So total number of ways to get a flush is $\binom{10}{2}-2$. \newline
Let B be an event that flush is drawn.
Then probability of it happening is: \begin{equation}
    P(A) =\frac{\binom{10}{2}-2}{\binom{47}{2}}
\end{equation}
\subsection*{Question 4, Section 2, Chapter 2 (\cite{toi3rd})}

\noindent\textbf{Assigned to Akbota Assainova}\\

\noindent Firstly, let's discuss about the cases of full house or four-of-a-kind to solve this problem. 
\newline 
\textbf{Four of kind:} 
a hand where four of the cards are with the same rank and another one with different rank. \textit{Example:} $K\varheartsuit, K\spadesuit, K\clubsuit, K\vardiamondsuit, A\vardiamondsuit$
\newline
\textbf{full house:} 
 a hand of 5 cards representing two cards with the same rank and another three cards with another same rank. \textit{Example:} $Q\varheartsuit, Q\spadesuit, Q\clubsuit, K\spadesuit, K\clubsuit$ 
\newline
Now, we can solve the problem itself.
\newline
There are 52 cards in a deck, 4 of each type.
The poker playeralina.abdra16@gmail.com has 5 cards, leaving 47 in the deck.
To get a full house he needs one more ace. there are 3 left in the deck.
To get a 4 of a kind he needs one more queen. there is only one left in the deck.
\newline
There are 3 aces, so three of those circumstances. there are 47 cards remaining in the deck, so there is a total of 47 circumstances. the probability of a full house is $P_{full house}=\frac{3}{47}$.
\newline
The same thing applies to the four-of-a-kind. There is only 1 queen left, so just one of those circumstances, and there are 47 cards, so 47 possible circumstances. the probability of getting a four-of-a-kind is $P_{four-of-a-kind}=\frac{1}{47}$
\newline
Thus, the final probability is: 
\newline 
P(draws to full house or four-of-a-kind)$=P_{full-house} + P_{four-of-a-kind}$
\newline
P(draws to full house or four-of-a-kind)$=\frac{3}{47}+\frac{1}{47}=\frac{4}{47}$


\subsection*{Question 5, Section 2, Chapter 2 (\cite{toi3rd})}

\noindent\textbf{Assigned to Dariya Kalymova}\\

\noindent A bridge hand (thirteen cards) is dealt from a stan dard 52-card deck. Let A be the event that the hand contains four aces; let B be the event that the hand contains four kings. Find
\begin{math}
 P(A \cup B).
 \end{math}
\newline
\textit{\textbf{Solution:}}
\begin{math}
 P(A) = \frac{\binom{4}{4} \binom{48}{9}}{\binom{52}{13}}.\newline 
  P(B) = \frac{\binom{4}{4} \binom{48}{9}}{\binom{52}{13}}.\newline
   P(A \cap B) = \frac{\binom{4}{4} \binom{4}{4} \binom{44}{5}}{\binom{52}{13}}.\newline
  P(A \cup B) = P(A) + P(B)- P(A \cap B) = \frac{4 669 920}{1512}
\end{math}

\newpage
\printbibliography
\end{document}
